\chapter{Perspectives et conclusions}
% auteur : Fabien
% relu par : Clément Florian Oualid

Des perspectives de développement demeurent, à savoir~:
\begin{itemize}
    \item un système de chat privé (discussion instantanée et/ou message avec sujet) entre les utilisateurs afin de discuter des services qu'ils proposent~;
    \item l'implémentation du compte premium imaginé n'a pas été utilisé finalement dans le projet (système de gain en monnaie libre contre des euros, par exemple)~;
    \item la mise en place d'avatars qui pourront être utiles notamment pour le forum et le chat privé.
\end{itemize}

Les kiwis (mini-avatars représentatifs de l'échelon social de l'utilisateur sous forme de poivriers ou de grains de poivre par exemple), initialement prévus n'ont pas vus le jour et sont donc implémentables dans un futur proche. Aussi, une barre latérale présentant les services qui sont proposés par les utilisateurs possédant des hauts karmas pourrait être implémentée sur les vues liées aux services.

Finalement, la majeure partie des idées énoncées lors des premières réunions de projet ont été réalisées, le diagramme des cas d'utilisation ainsi que le diagramme de classe initiaux ont globalement été bien suivis.
En effet, un utilisateur peut créer des services en échange de la monnaie libre choisie, il peut gérer ses services et en réserver d'autres.
Il peut aussi évaluer un service réalisé par un prestataire ce qui donne des points de karma au prestataire, de ce fait, il gagne en «~confiance~» et ses services en sont valorisés et deviennent attrayants.
Un utilisateur peut aussi créer un groupe, rejoindre un groupe et créer des services associés à un groupe, ainsi, les services internes au groupe sont accessibles plus facilement par les membres du groupes et peuvent être catégorisés selon le désir du fondateur du groupe.
Aussi, plusieurs types de services sont disponibles, à savoir, les ventes, le couchsurfing, le covoiturage ainsi que d'autres services que nous considérons comme des services de base (cours d'anglais par exemple).
Un forum est également disponible afin de communiquer entre membres de Poavre, le forum fonctionne par topics et par commentaires de topics. Les topics sont classés par ratio (likes/dislikes) afin de faire remonter les plus intéressants.
Un espace administrateur est disponible afin de gérer le site et ses diverses fonctionnalités.
