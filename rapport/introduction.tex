\chapter{Introduction}

\section{Le sujet}
% auteur : Adrien
% relu par : Clément, Fabien, Thin

\subsection{Objectif de ce projet}

L'objectif de ce projet est la réalisation d'un site Web social permettant à des personnes de proposer des services, d'obtenir des services et d'échanger en utilisant des monnaies libres.

\subsection{Ce qu'est une monnaie libre}

Une monnaie libre est un concept économique inspiré des règles fondamentales du logiciel libre (voir \cite{monnaie}).
Pour être considérée comme libre, une monnaie propose :
\begin{itemize}
    \item la liberté du choix du système monétaire (la monnaie ne s'impose pas)~;
    \item la liberté d'utilisation des ressources (économiques et monétaires)~;
    \item la liberté d'estimation et de production de toute valeur~;
    \item la liberté d'échanger dans la monnaie (afficher, comptabiliser, échanger dans l'unité monétaire choisie).
\end{itemize}

\subsection{Les SELs (réseaux sociaux autour d'une monnaie libre)}


Un SEL (Système d'Échange Local) (voir \cite{sel}) est un système de partage de biens et services entre membres d'une même communauté et fondé autour d'une monnaie libre.
Ces systèmes fonctionnent considérablement sur la confiance des membres de cette communauté envers la monnaie et envers leurs pairs, la localité de ces communautés aidant.

Il est fréquent que les comptes y soient tenus dans des carnets, ne soient pas à jour ou que les échanges finissent par se faire sans paiement entre connaissances.

\section{Cahier des charges}
% auteur : Adrien (copié-collé de http://www.lirmm.fr/~ferber/TER/web_social.htm ), Florian
% relu par : Clément, Fabien,Thin

\subsection{Fonctionnalités attendues}

Un certain nombre de fonctionnalités étaient à l'origine prévues par le cahier des charges, ainsi qu'un certain nombre de contraintes. Bien que toutes n'aient pas été implémentées, elles ont été la base de notre travail~:
\begin{itemize}
    \item intégration à un groupe, définition d'un profil (photos, etc...)~;
    \item définition de ses intérêts~;
    \item définir les services que l'on propose, leur donner un prix dans la monnaie libre associée au groupe (ou dans une autre monnaie)~;
    \item regroupement par intérêts~;
    \item évaluation des services proposés~;
    \item communication par sujets ou petits forums avec soutien (et remontée des informations les plus appréciées avec des techniques semblables à Reddit)~;
    \item partage d'informations...
\end{itemize}

\subsection{Contraintes}
Les contraintes étant~:
\begin{itemize}
    \item utiliser un framework ou CMS existant~;
    \item l'évaluation du logiciel de base devra être réalisée~;
    \item la possibilité de tester ce logiciel dans un groupe de personnes fonctionnant dans un JEU (Jardin d'Echange Universel), un système de monnaie libre et d'échange qui existe dans plusieurs régions, dont Montpellier.
\end{itemize}

